\chapter{Introduction}
\label{ch:Introduction}

With the increasing use of the Internet in education, large amounts of educational data are created. Hence, it is important to be able to develop new tools to analyze this kind of data because all this information can be used to better understand the learning process of students. It is then, one of today's biggest challenges to educational institutions, the exponential growth of educational data and how to transform all of this data into new insights for the benefit of students, teachers, and administrators \cite{romero2020educational}.

In this environment, Learning Analytics\abrv{Learning Analytics}{LA}(LA) can be seen as an important area of research. LA combines learning, often defined as the process of acquiring competence and understanding, with analysis methods to reveal patterns from the educational data \cite{khalil2015learning}. The Society for Learning Analytics Research defines it as "the measurement, collection, analysis and reporting of data about learners and their contexts, for purposes of understanding and optimizing learning and the environments in which it occurs \cite{solar11}.

The main goal of LA is to be able to extract information from educational data to support decision-making in educational-related business. Thus, several factors work in favor of LA's popularity, such as (1) enthusiasm in employing a data-driven approach to make better decisions, similar to what happens in business intelligence; (2) it already exists good statistical, machine learning, and data mining techniques to look for patterns in data that can be effortlessly adapted to educational data; (3) data generation, storage, and processing are relatively easy with current computer capacity; (4) universities benefit from having reduced dropout rates and improved course quality \cite{linan2015educational}. 

Some common applications of LA are: (1) prediction - where the goal is to infer some target variable from some combination of other variables; (2) clustering - aims to identify groups of similar characteristics; (3) relationship mining - studies the relationships among variables; among others. The focus of this work is on the prediction method, seeing that academic performance prediction is one of the most popular subjects in the field of LA \cite{akccapinar2019using}.

In addition to LA, it also exists other communities with joint interest such as Educational Data Mining\abrv{Educational Data Mining}{EDM}(EDM). LA and EDM are both fields that share a common interest in data-intensive approaches to educational research to enhance the educational process. The difference is that LA focuses on the educational side and EDM focuses on the technological side. LA concentrates on data-driven decision-making and combining the technical and pedagogical dimensions of learning by putting into use known predictive models. In contrast, EDM is usually searching for new patterns in data and developing new algorithms and/or models \cite{romero2020educational}. In this work, the procedures of EDM are emphasized.

\section{Motivation}

The Instituto Metrópole Digital\abrv{Instituto Metrópole Digital}{IMD}(IMD) offers semiannually a technical course in Informational Technology\abrv{Informational Technology}{IT}(IT)\footnote[1]{\hspace{1mm}Course description available in: \url{https://portal.imd.ufrn.br/portal/tecnico}}. This course has classes that are executed in a hybrid learning model, with tasks being done in a virtual learning environment, but also with weekly encounters in a traditional classroom. In this learning model, the student has activities to be done every week. All of those activities are graded and, as result, the student's final score in the course is determined by a combination of those grades received in each class throughout the course schedule.

Given how long the course lasts and how numerous grades the student receives, it can be difficult, for both students and teachers, to visualize how well the student is performing, at any given week. Moreover, it can also be interesting if students and teachers can see the likelihood that each student has of passing the course.

In short, the following questions are what motivated this research to be done: (1) can the students and teachers of the IMD have a better visualization of how good or bad the students' performance currently is, in comparison to current classmates and previous students; and also: (2) how good can the students' final course result be predicted using data of the early weeks of the course.

\section{Objectives}

This work's main goal is to find a model of student success prediction, applied to the data of the IT course offered by the IMD. This model will then be used by a web system developed to help students and teachers receive periodically feedback of each student's current academic performance, signaling those students that are currently underperforming.

To achieve the main goal, the sub-goals are:

\begin{enumerate}
    \item Research the common machine learning algorithms that can be applied to the data.
    \item Comparative analysis of the performance of those algorithms.
    \item Develop a web system prototype for monitoring student performance.
\end{enumerate}

\section{Structure}

The next chapter presents the background information with the main topics addressed in this work. Chapter \ref{ch:Methodology} presents the methodology of this research. Chapter \ref{ch:Experiments} displays the experiments made and the results obtained. Finally, Chapter \ref{ch:Conclusions} gives the final considerations and possible future works.