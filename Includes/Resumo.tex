% Resumo em língua vernácula
\begin{center}
	{\Large{\textbf{Developing a web system for predicting student success using learning analytics}}}
\end{center}

\vspace{1cm}

\begin{flushright}
	Autor: Lucas Aléssio Anunciado Silva\\
	Orientador(a): Dra. Isabel Dillmann Nunes
\end{flushright}

\vspace{1cm}

\begin{center}
	\Large{\textsc{\textbf{Resumo}}}
\end{center}

\noindent Obter informação, a partir de dados educacionais, que seja beneficial a tomada de decisão, em contexto acadêmico, é o que Learning Analytics (LA) se interessa. Logo, nesse trabalho, dados de notas e frequência em um curso híbrido é usado para pesquisar se o resultado dos alunos pode ser previsto nas semanas iniciais do calendário acadêmico. A fonte dos dados é o curso técnico em Tecnologia da Informação (TI), ofertado pelo Instituto Metrópole Digital (IMD), em turmas entre 2015 e 2020. A pesquisa foi conduzida seguindo os passos do ciclo do LA, comparando quatro algoritmos de classificação comuns. Os resultados indicam que o modelo binário de Regressão Logística previu corretamente aqueles estudantes que devem passar no curso com uma taxa que variou entre 66,51\%, na semana 1, e 88,67\%, na semana 18. A partir desse modelo, um protótipo de aplicação \emph{web} foi desenvolvido para atuar como um sistema de pré-aviso para professores e estudantes. Esse sistema pode ser usado em intervenções pontuais para aprimorar a desempenho dos estudantes, desta maneira, reforçando o processo de aprendizado.

\noindent\textit{Palavras-chave}: \emph{learning analytics}, predição de performance acadêmica, sistema de pré-aviso.