% Resumo em l ngua estrangeira (em ingl s Abstract, em espanhol Resumen, em franc s R sum )
\begin{center}
	{\Large{\textbf{Developing a web system for predicting student success using learning analytics}}}
\end{center}

\vspace{1cm}

\begin{flushright}
	Author: Lucas Aléssio Anunciado Silva\\
	Advisor: Isabel Dillmann Nunes, PhD
\end{flushright}

\vspace{1cm}

\begin{center}
	\Large{\textsc{\textbf{Abstract}}}
\end{center}

\noindent Obtaining information from educational data that is beneficial to decision-making in an academic context is what Learning Analytics (LA) is concerned about. Therefore, in this work, data from grades and attendance in a hybrid course is used to research whether the students' finals results could be predicted in the early weeks of the course schedule. The source of the data is the Informational Technology (IT) technical course, offered by Instituto Metrópole Digital (IMD), in classes between 2015 and 2020. The research was conducted following the steps of the LA cycle, comparing four commonly used classification algorithms. The results indicated that the Logistic Regression binary model accurately predicted those students that would pass the course with a rate that ranged from 66.51\%, in week 1, to 88.67\%, in week 18. Given this model, a web system prototype was developed to act as an early warning system to teachers and students. This system could be used for timely intervention to improve student performance, thus, enhancing the educational learning process.

\noindent\textit{Keywords}: learning analytics, academic performance prediction, early warning system.